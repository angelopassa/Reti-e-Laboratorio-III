\section{Commutazione e Ritardi}

    \begin{definition}[Commutazione]
        Modalità con cui viene determinato il percorso sorgente-destinazione e vengono dedicate ad esso le risorse della rete.
    \end{definition}

    Esistono due meccanismi:
    \begin{itemize}
        \item \textcolor{purple}{Commutazione di Circuito}
        \item \textcolor{purple}{Commutazione di Pacchetto}
    \end{itemize}

    \paragraph{Metriche}Data una comunicazione tra due host, indipendentemente dal tipo di commutazione utilizzata, è possibile utilizzare delle metriche comuni per l'analisi delle prestazioni:
    \begin{itemize}
        \item \textcolor{purple}{Bandwidth:} larghezza dell’intervallo di frequenze utilizzato dal sistema trasmissivo (si misura in Hz).
        \item \textcolor{purple}{Transmission Rate:} quantità di dati (bits) che possono essere trasmessi per unità di tempo su un certo collegamento (dipende dal bandwidth ma anche dal mezzo trasmissivo, rumore, ecc...).
        \item \textcolor{purple}{Throughput:} quantità di dati che possono essere trasmessi con successo dalla sorgente alla destinazione in un certo intervallo di tempo al netto di perdite sulla rete (duplicazioni, protocolli, ecc...).
        \newline
        In un percorso da una sorgente a una destinazione un pacchetto può passare attraverso numerosi link, perciò il throughput dell'intero percorso è dato dal percorso intermedio con throughput minore.
    \end{itemize}

    %capire come mettere il < correttamente
    NB: Throughput < Transmission Rate
        

    \subsection{Commutazione di Circuito}   
    Nella commutazione di circuito si instaura un cammino dedicato tra i due dispositivi che vogliono comunicare. Il percorso viene stabilito all’inizio della comunicazione (\textcolor{purple}{Setup}) e vengono dedicate risorse alla comunicazione (canale logico o circuito) in modo esclusivo. Le risorse allocate sono garantite per tutta la durata della comunicazione, indipendentemente dall’utilizzo effettivo.
    \paragraph*{Canale logico} Per quanto riguarda l'assegnazione di un canale di comunicazione logico esistono due principali metodi:
        \begin{itemize}
            \item \textbf{\textcolor{purple}{FDM}}: Frequency Division Multiplexing, il canale di comunicazione viene suddiviso in bande di frequenze ognuna delle quali viene assegnata in modo esclusivo ad una certa connessione.
            \item \textbf{\textcolor{purple}{TDM}}: Time Division Multiplexing, il tempo viene suddiviso in slot di tempo. Ogni comunicazione ha uno o più slot periodici assegnati nei quali può trasmettere alla velocità massima del canale.
        \end{itemize}

    \paragraph*{Vantaggi:}
    \begin{itemize}
        \item Performance garantite.
        \item Tecnologie di switching efficienti.
    \end{itemize}

    \paragraph*{Svantaggi:}
    \begin{itemize}
        \item Necessaria una fase di instaurazione della comunicazione.
        \item le risorse rimangono inattive se non utilizzate (non c’è condivisione).
    \end{itemize}


    \subsection{Commutazione di Pacchetto}
    Nella commutazione di pacchetto il flusso di dati punto-punto viene suddiviso in pacchetti. Ogni pacchetto è instradato singolarmente e indipendentemente dagli altri pacchetti della stessa comunicazione (possono seguire lo stesso percorso o percorsi diversi in base alle necessità).
    \newline
    I commutatori (es. router) devono ricevere integralmente i pacchetti prima di poterne continuare l'instradamento. Quando ci sono più pacchetti in ingresso ripetto a quanto il canale può supportare, questi vengono messi in coda in dei buffer (se il buffer è pieno i pacchetti vengono persi).
    
    \paragraph*{Vantaggi:}
    \begin{itemize}
        \item Risorse trasmissive usate solo se necessario.
        \item Fase di setup e segnalazione della connessione non richiesta.
    \end{itemize}

    \paragraph*{Svantaggi:}
    \begin{itemize}
        \item Tecnologie di inoltro poco effienti (calcolo del percorso indipendente per ogni pacchetto).
        \item Ritardi variabili nel percorso end-to-end (jitter).
        \item protocolli necessari per un trasferimento dati affidabile, controllo della congestione.
    \end{itemize}

    \subsection{Ritardi}

    \begin{definition}[Latenza] 
        Tempo richiesto affinché un messaggio arrivi a destinazione dal momento in cui il primo bit parte dalla sorgente.
    \end{definition}

    Il valore della latenza in una rete a commutazione di pacchetto è determinato da 4 tipologie di ritardo:
    \begin{itemize}
        \item \textcolor{purple}{Ritardo di eleborazione}, dovuto ai controlli di errore sui bit e dal calocolo del percorso di uscita del pacchetto.
        \item \textcolor{purple}{Ritardo di accodamento}, dovuta all'attesa dei pacchetti nel buffer la quale dipende dal tipo e dall'intensità del traffico.
        \item \textcolor{purple}{Ritardo di trasmissione:}, tempo impiegato per trasmettere un pacchetto sul canale. 
        \newline Dipende da due valori
        \begin{itemize}
            \item \textcolor{blue}{R}, rate di trasmissione sul canale (in bps).
            \item \textcolor{blue}{L}, lungezza del pacchetto (in bit).
        \end{itemize}
        Si calcola come \textcolor{blue}{L/R}.
        \item \textcolor{purple}{Ritardo di propagazione}, tempo impiegato da 1 bit per essere propagato da un nodo all’altro
        \newline Dipende anch'esso da due valori
        \begin{itemize}
            \item \textcolor{blue}{d}, lunghezza del collegamento fisico (in metri).
            \item \textcolor{blue}{s}, velocità di propagazione del mezzo.
        \end{itemize}
        Si calcola come \textcolor{blue}{d/s}.
    \end{itemize}
    
    La latenza è quindi la somma di tutti questi 4 ritardi. Nella pratica però i ritardi di elaborazione e accodamento vengono trascurati e la latenza end-to-end è data dalla somma dei ritardi di propagazione e trasmissione di tutti i collegamenti intermedi.